\documentclass[a4paper,11pt]{article}
\usepackage[UTF8]{ctex}
\usepackage[top=0.3in, bottom=0.5in, left=0.5in, right=0.5in]{geometry}
\usepackage{enumitem}
\usepackage[hidelinks]{hyperref}
\usepackage{setspace}

\setstretch{1.15}
\setlist{itemsep=3pt,topsep=4pt,parsep=0pt,partopsep=0pt}

% Add space between sections
\usepackage{titlesec}
\titlespacing*{\section}{0pt}{14pt}{8pt}

% Add space between subsections within sections
\newcommand{\spacedsection}[2]{%
    \vspace{10pt}%
    \textbf{#1} \hfill #2 \\%
}

% Add space between project entries
\newcommand{\projectspace}{\vspace{8pt}}


\begin{document}

\centerline{\Large\bf 张启航}
\centerline{个人主页: \href{https://zhangsetsail.com/zn_ch}{zhangsetsail.com} \\ Email: qihangzhang@link.cuhk.edu.cn \\ Github: \href{https://github.com/Zhang-Setsail}{Zhang-Setsail}} 

\section*{Education}

\textbf{香港中文大学(深圳)} \hfill 2023/08 - 2026/06 \\
学术型硕士
\begin{itemize}[noitemsep]
  \item 专业:计算机科学
  \item 导师:\href{https://scholar.google.com/citations?hl=zh-CN&user=igqPS8sAAAAJ&view_op=list_works&sortby=pubdate}{孙启霖}教授
  \item 研究方向:计算成像与低层视觉
\end{itemize}

\projectspace
\textbf{东京工业大学} \hfill 2025/04 - 2025/09 \\
科研助理
\begin{itemize}[noitemsep]
  \item 指导老师:\href{https://www.vip.sc.eng.isct.ac.jp/mtanaka/}{田中正行}教授和\href{http://www.ok.sc.e.titech.ac.jp/~ymonno/}{紋野雄介}教授
  \item 研究方向:偏振成像
\end{itemize}

\projectspace
\textbf{香港中文大学(深圳)} \hfill 2019/08 - 2023/06 \\
工学学士
\begin{itemize}[noitemsep]
  \item 专业:计算机科学与工程
\end{itemize}

\section*{Work Experience}
% \textbf{香港中文大学(深圳)数据科学学院} \hfill 2023/09 - 至今 \\
%  助教 \hfill 深圳
% \begin{itemize}[noitemsep]
% \item 担任Database System(CSC3170)、Parallel Programming(CSC4005)和Computer Lab(CSC1002)等核心课程的助教,负责课程支持与学生指导工作。
% \item 优化课程作业模板与自动化评分系统,引入前沿技术(\href{https://github.com/tonyyxliu/CUHKSZ-CSC4005/tree/main/project4}{如Triton框架})作为编程实践工具,提升学生实践能力与课程实用性。
% \end{itemize}

\textbf{Vivo移动通信} \hfill 2022/06 - 2022/11 \\
 测试引擎开发实习生 \hfill 深圳
\begin{itemize}[noitemsep]
\item 参与Android测试引擎的开发与维护,编写并优化核心代码,开发多个基于Python的ADB自动化测试组件以提升测试效率与覆盖率,同时提高代码的可读性和可维护性。
\item 扩展测试引擎功能,支持多国定制化测试需求,确保测试流程适配不同地区的语言和文化要求。
\item 实现Google Android测试基准的全流程自动化,编写脚本并集成工具链,确保测试流程符合Google测试工程规范。
\item 编写并更新测试引擎的使用文档,帮助测试开发团队快速上手,减少沟通成本并提高团队协作效率。
\end{itemize}

\section*{Research Experience}
\textbf{彩色偏振图像联合去噪与去马赛克} \hfill 2025/04 - 2025/09 \\
Collaborators:\href{https://www.vip.sc.eng.isct.ac.jp/mtanaka/}{田中正行}教授,\href{http://www.ok.sc.e.titech.ac.jp/~ymonno/}{紋野雄介}教授
\begin{itemize}[noitemsep]
\item 提出了首个面向彩色偏振成像的联合去噪与去马赛克重建框架,在共享特征空间内协同建模并恢复退化图像。
\item 设计了基于编码器特征融合机制的图像重建模型,显著提升了重建质量。
\item 构建了首个真实世界彩色偏振图像数据集,支持模型训练与评估。
\item 在多个指标上显著优于现有偏振图像去噪与去马赛克方法。
\end{itemize}

\projectspace
\textbf{超高动态范围传感器快速Tonemapping和ISP} \hfill 2024/05 - 2024/11 \\
Collaborators:\href{https://scholar.google.com/citations?hl=zh-CN&user=igqPS8sAAAAJ&view_op=list_works&sortby=pubdate}{孙启霖}教授,\href{https://www.pointspread.cn/}{点昀技术}
\begin{itemize}[noitemsep]
\item 设计并实现基于硬件加速的高速色调映射算法,支持24Bit-HDR图像的实时处理与LDR显示,实现动态范围接近人眼感知极限(130-140dB)视频的LDR输出。
\item 开发端到端可学习的ISP算法,通过小型神经网络实时估算Global Tonemapping参数,显著提升24位视频流的处理效率与图像质量。
\item 在ISP算法中引入端到端可微设计,支持任务扩展与联合优化;例如,将ISP与目标检测任务结合,优化ISP输出以适配检测网络,以提升目标检测精度。
\item 通过实验验证算法在多种场景下的性能,确保其在复杂光照条件下的稳定性。
\end{itemize}

\projectspace
\textbf{RGB-IR传感器ISP和快速反射去除} \hfill 2023/08 - 2023/12 \\
Collaborators:\href{https://scholar.google.com/citations?hl=zh-CN&user=igqPS8sAAAAJ&view_op=list_works&sortby=pubdate}{孙启霖}教授, \href{https://www.vip.sc.eng.isct.ac.jp/mtanaka/}{田中正行}教授,\href{http://www.ok.sc.e.titech.ac.jp/~ymonno/}{紋野雄介}教授
\begin{itemize}[noitemsep]
\item 设计并实现基于GPU加速的RGB-IR传感器图像信号处理管线(ISP),显著提升图像处理效率,满足实时性要求。
\item 提出一种基于红外光特性的反射去除算法,利用NIR波段对玻璃反射的低敏感性,结合可见光波段信息,有效分离并去除反射干扰,提升图像质量。
\item 采用引导滤波技术优化RGB-IR图像的特征提取与融合过程,在保证算法精度的同时,大幅降低模型复杂度与参数量,提升计算效率。
\end{itemize}

\section*{Project Experience}
\projectspace
\textbf{相机仿真图像生成管线} \hfill 2024/08 - 至今 \\
Collaborators:\href{https://scholar.google.com/citations?hl=zh-CN&user=igqPS8sAAAAJ&view_op=list_works&sortby=pubdate}{孙启霖}教授,实验室成员(负责光学部分),华为(横向委托方)
\begin{itemize}[noitemsep]
\item 设计并实现了从光学到传感器的完整图像仿真处理流程。
\item 使用空间可变的点扩散函数(PSF)卷积核模拟镜头退化。
\item 基于传感器噪声特性构建噪声模型并生成噪声Map。
\item 完成了可扩展、可移植的仿真管线,可适配不同应用场景,为算法开发与验证提供高保真数据支撑。
\end{itemize}

\projectspace
\textbf{CPU/GPU部分算子并行计算实现(MPI, OpenMP, CUDA, Triton)} \hfill 2024/09 - 2024/12 \\
Collaborators:\href{https://github.com/EnderturtleOrz}{徐源}
\begin{itemize}[noitemsep]
\item 构建模块化CPU/GPU并行算子库,实现部分传统图像处理算子(灰度化、模糊、Sobel、双边滤波)、矩阵乘法及部分深度神经网络卷积/全连接层,并支持反向传播。
\item 使用MPI、Pthreads、OpenMP实现CPU上的并行优化,使用CUDA和Triton实现GPU上的并行优化。
\item GPU/Triton实现的图像算子和矩阵乘法在相同条件下性能优于基线-PyTorch实现。
% \item 项目代码:\href{https://github.com/tonyyxliu/CUHKSZ-CSC4005}{作为TA提供模版与Baseline}
\end{itemize}

\projectspace
\textbf{基于HDR-Plus的HDR视频处理应用开发} \hfill 2022/12 - 2023/03
\begin{itemize}[noitemsep]
\item 基于谷歌HDR+论文,实现多帧短曝光Raw合成的高动态范围(HDR)视频处理管线。
\item 使用多帧对齐与融合、滤波降噪和局部色调映射等HDR+提出的ISP模块。
\item 使用Python和PyQt5框架开发了图形用户界面(GUI)应用,方便用户操作,支持选择RAW图像序列进行处理,并通过调用FFmpeg将处理后的图像帧合成为最终的HDR视频。
% \item 项目页面:\href{https://github.com/Zhang-Setsail/Computer_Graphics/tree/main/code/Project}{GitHub代码链接}
\end{itemize}

\section*{Research Interest}
\begin{itemize}[noitemsep]
\item \textbf{应用方向:} 图像处理、HDR成像、新型图像传感器
\item \textbf{技术方向:} 计算成像、低层视觉
\end{itemize}

\section*{Language and Skills}
\begin{itemize}[noitemsep]
\item \textbf{语言能力}:IELTS 7.0(阅读8.5 / 听力7.5 / 口语6.0 / 写作6.0)能够熟练阅读英文技术文档并交流。
\item \textbf{编程技能}:掌握Python、C++等编程语言,具备扎实的编程基础和良好的代码实践能力。
\item \textbf{技术专长}:熟悉传统图像处理算法和基于深度学习的图像处理技术,深入了解图像信号处理管线(ISP)的各个模块及其实现原理。
\item \textbf{项目信息}:部分可公开项目代码与详细内容请\href{https://zhangsetsail.com/}{查看此Link}。
\end{itemize}

\end{document}