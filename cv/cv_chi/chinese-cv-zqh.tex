\documentclass[a4paper]{article}
\usepackage[UTF8]{ctex}
\usepackage[top=0.3in, bottom=0.5in, left=0.5in, right=0.5in]{geometry}
\usepackage{enumitem}
\usepackage[hidelinks]{hyperref}
\usepackage{setspace}

\setstretch{1.1}
\setlist{itemsep=2pt,topsep=2pt,parsep=0pt,partopsep=0pt}

% Add space between sections
\usepackage{titlesec}
\titlespacing*{\section}{0pt}{12pt}{6pt}

% Add space between subsections within sections
\newcommand{\spacedsection}[2]{%
    \vspace{8pt}%
    \textbf{#1} \hfill #2 \\%
}


\begin{document}

\centerline{\Large\bf 张启航}
\centerline{个人主页: \href{https://zhangsetsail.com/zn_ch}{zhangsetsail.com} \\ Email: qihangzhang@link.cuhk.edu.cn \\ Github: \href{https://github.com/Zhang-Setsail}{Zhang-Setsail}} 

\section*{Education}
\textbf{香港中文大学(深圳)} \hfill 2019/08 - 2023/06 \\
工学学士
\begin{itemize}[noitemsep]
  \item 专业:计算机科学与工程
\end{itemize}

\textbf{香港中文大学(深圳)} \hfill 2023/08 - 至今 \\
理学硕士
\begin{itemize}[noitemsep]
  \item 专业:计算机科学
\end{itemize}

\section*{Work Experience}
\textbf{香港中文大学(深圳)数据科学学院} \hfill 2023/09 - 至今 \\
 助教 \hfill 深圳
\begin{itemize}[noitemsep]
\item 担任Database System和Parallel Programming等核心课程的助教,负责课程支持与学生指导工作。
\item 优化课程作业模板与自动化评分系统,引入前沿技术(\href{https://github.com/tonyyxliu/CUHKSZ-CSC4005/tree/main/project4}{如Triton框架})作为编程实践工具,提升学生实践能力与课程实用性。
\end{itemize}

\textbf{Vivo移动通信} \hfill 2021/07 - 2021/10 \\
 软件开发实习生 \hfill 深圳
\begin{itemize}[noitemsep]
\item 参与Android测试引擎的开发与维护,编写并优化核心代码,开发多个基于Python的自动化测试组件以提升测试效率与覆盖率,同时提高代码的可读性和可维护性。
\item 扩展测试引擎功能,支持多国定制化测试需求,确保测试流程适配不同地区的语言和文化要求。
\item 实现Google Android测试基准的全流程自动化,编写脚本与工具链集成,确保测试流程符合Google测试工程规范。
\item 编写并更新测试引擎的使用文档,帮助测试团队快速上手,减少沟通成本并提高团队协作效率。
\end{itemize}

\section*{Project Experience}
\textbf{超高动态范围传感器快速Tonemapping和ISP} \hfill 2024/04 - 至今
\begin{itemize}[noitemsep]
\item 设计并实现基于硬件加速的高速色调映射算法,支持24Bit-HDR图像的实时处理与LDR显示,动态范围接近人眼感知极限(130-140dB)。
\item 开发端到端可学习的ISP算法,通过小型神经网络实时估算Global Tonemapping参数,显著提升24位视频流的处理效率与图像质量。
\item 在ISP算法中引入端到端可微设计,支持任务扩展与联合优化;例如,将ISP与目标检测任务结合,优化ISP输出以适配检测网络,显著提升目标检测精度。
\item 通过实验验证算法在多种场景下的性能,确保其在复杂光照条件下的稳定性。
\end{itemize}

\textbf{RGB-IR传感器ISP和快速反射去除} \hfill 2023/08 - 2023/12
\begin{itemize}[noitemsep]
\item 设计并实现基于GPU加速的RGB-IR传感器图像信号处理管线(ISP),显著提升图像处理效率,满足实时性要求。
\item 提出一种基于红外光特性的反射去除算法,利用NIR波段对玻璃反射的低敏感性,结合可见光波段信息,有效分离并去除反射干扰,提升图像质量。
\item 采用引导滤波技术优化RGB-IR图像的特征提取与融合过程,在保证算法精度的同时,大幅降低模型复杂度与参数量,提升计算效率。
\end{itemize}

\textbf{基于HDR-Plus的HDR视频处理应用开发} \hfill 2022/12 - 2023/02
\begin{itemize}[noitemsep]
  \item 基于HDR-Plus文章,应用文章中多帧融合,滤波降噪等新提出的ISP模块
  \item 使用PyQt实现HDR-Plus处理序列Raw数据并生成HDR视频的GUI应用[\href{https://github.com/Zhang-Setsail/Computer_Graphics/tree/main/code/Project}{Link}]
\end{itemize}

\section*{Language and Skills}
\begin{itemize}[noitemsep]
\item \textbf{语言能力}:IELTS 7.0(阅读8.5 / 听力7.5 / 口语6.0 / 写作6.0)能够熟练阅读英文技术文档并交流。
\item \textbf{编程技能}:掌握Python、C++等编程语言,具备扎实的编程基础和良好的代码实践能力。
\item \textbf{技术专长}:熟悉传统图像处理算法和基于深度学习的图像处理技术,深入了解图像信号处理管线(ISP)的各个模块及其实现原理。
\end{itemize}

\end{document}
